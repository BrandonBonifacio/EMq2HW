\documentclass[12pt,letterpaper]{hmcpset}
\usepackage[margin=1in]{geometry}
\usepackage{graphicx}
\usepackage[makeroom]{cancel}
\usepackage{array}
\usepackage{mathtools}
\usepackage{marginnote}
\usepackage{units}
\usepackage{xfrac}
\usepackage{enumerate}
\usepackage{amsmath}
\usepackage{fancyhdr}
\usepackage{pgfplots}
\usepackage{hyperref}
\usepackage{nopageno}
\usepackage{boondox-cal}
\hypersetup{
	colorlinks=true,
	linkcolor=blue,
	filecolor=magenta,
	urlcolor=magenta,
}

% info for header block in upper right hand corner
\name{} %put your name here
\class{Physics 51 Section \hspace{3mm}} %put your section here
\assignment{Homework 12}
\duedate{Monday, November 16, 2020}

\begin{document}
	\begin{problem}[T1.4:]
		A radio station broadcasts at a frequency $\nu$ = 91.5 MHz with a total radiated power of $P$ = 20 kW.
		\begin{enumerate}[(a)]
			\item What is the wavelength $\lambda$ of this radiation?
			\item What is the energy of each photon in eV? How many photons are emitted each second? How many photons are emitted each cycle?
			\item A particular radio receiver requires 2.0 microwatts of radiation to provide intelligible reception. How many 91.5 MHz photons does this require per second? per cycle?
			\item Do the answers to (b) and (c) indicate that the granularity of the electromagnetic radiation can be neglected in these circumstances?
		\end{enumerate}
	\end{problem}
	\clearpage



	\begin{problem}[T1.9:]
		A beam of UV light of wavelength $\lambda$ = 197.0 nm falls onto a metal cathode. The stopping potential needed to keep any electrons from reaching the anode is 2.08 V.
		\begin{enumerate}[(a)]
			\item What is the work function $W$ of the cathode surface, in ev?
			\item What is the velocity $v$ of the fastest electrons emitted from the cathode? \textit{Note}: Since $K_{max}/mc^2 \ll 1$, the nonrelativistic expression for the kinetic energy can be utilized here.
			\item If Avogadro's number of photons strikes each square meter of the surface in one hour, what is the average intensity $I$ of the beam, in units of W/m²?
		\end{enumerate}
	\end{problem}
	\clearpage



	\begin{problem}[T1.13:]
		The maximum kinetic energy of electrons ejected from sodium is 1.85 eV for radiation of 300 nm and 0.82 eV for radiation of 400 nm. Use this data to determine Planck's constant and the work function of sodium.
	\end{problem}
	\clearpage



	\begin{problem}[T1.19:]
		Express the complex number $z_1 = \frac{\sqrt{3} + i}{2}$ in the form $re^{i\phi}$. What about $z_1 = \frac{1 + \sqrt{3}i}{2}$?
		If these complex numbers are the probability amplitudes for photons to be detected, what is the probability in each case?
	\end{problem}
	\clearpage
\end{document}