\documentclass[12pt,letterpaper]{hmcpset}
\usepackage[margin=1in]{geometry}
\usepackage{graphicx}
\usepackage[makeroom]{cancel}
\usepackage{array}
\usepackage{mathtools}
\usepackage{marginnote}
\usepackage{units}
\usepackage{xfrac}
\usepackage{enumerate}
\usepackage{amsmath}
\usepackage{fancyhdr}
\usepackage{pgfplots}
\usepackage{hyperref}
\usepackage{nopageno}
\usepackage{boondox-cal}
\hypersetup{
	colorlinks=true,
	linkcolor=blue,
	filecolor=magenta,      
	urlcolor=magenta,
}

% info for header block in upper right hand corner
\name{} %put your name here
\class{Physics 51 Section \hspace{3mm}} %put your section here
\assignment{Homework 9}
\duedate{Monday, October 26, 2020}

\begin{document}
	\begin{problem}[35P1:]
		A thin, plastic disk of radius $R$ has a charge $q$ uniformly distributed over its surface.
		If the disk rotates at an angular frequency $\omega$ about its axis, show that the magnetic dipole moment of the disk is
		\[\mu = \frac{\omega qR^2}{4}.\]
		(Hint: The rotating disk is equivalent to an array of current loops.)
	\end{problem}
	\clearpage



	\begin{problem}[35E12:]
		The dipole moment associated with an atom of iron in an iron bar is 2.22 $\mu_B$.
		Assume that all the atoms in the bar, which is 4.86 cm long and has a cross-sectional area of 1.31 cm², have their dipole moments aligned.
		\begin{enumerate}[(a)]
			\item What is the dipole moment of the bar?
			\item What torque must be exerted to hold this magnet at right angles to an external field of 1.53 T?
		\end{enumerate}
	\end{problem}
	\clearpage
\end{document}
